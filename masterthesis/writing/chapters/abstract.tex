
% Background
On large scales, comparable to the horizon, relativistic effects will affect the cosmological observables. In order to solve for these effects, one need to consistently solve for the metric, velocities and densities in a particular gauge. When simulating large-scale structures we use N-body simulations, which are usually performed in the Newtonian limit. However, it is not obvious that Newtonian gravity yield a good global description of an in-homogeneous cosmology across all scales ~\parencite{jeong_large-scale_2012}. However, literature suggest that Newtonian simulations are still solving the dynamics correctly, even on large scales close to the horizon where relativistic effects are important but may be corrected for ~\parencite{chisari_connection_2011} ~\parencite{green_newtonian_2012}.

% Literature results suggest that the relativistic corrections necessary on top of realistic Newtonian cosmologies should be very small~\parencite{chisari_connection_2011}. If this is the case, then this justifies the use of Newtonian simulations even on scales larger than the Hubble radius, whose results may be translated into relativistic cosmologies using relevant dictionaries~\parencite{green_newtonian_2012}.

Recently, \cite{adamek_gevolution_2016} developed a relativistic N-body code, \texttt{gevolution}, which evolves large scales structures based on the weak field expansion in GR. I investigate the differences in the gravitational dynamics between structures evolved with and without relativistic effects, with focus on the gravitational potential $\Phi$. This is a good choice for comparison as $\Phi$ is gauge invariant and the Newtonian and relativistic simulations are performed in different gauges.

The investigation is done by running 2000 simulations using identical $\Lambda$CDM cosmologies for the two gravity theories. The simulations are run using $64^4$ particles on a $256^3$ grid each with dimension $5120\text{ Mpc/h}$ which is a compromise in order to include both large and nonlinear scales. The data analysis consist of a preliminary analysis using conventional summary statistics, with focus on the bispectrum of $\Phi$. There is a difference in the two cases for low redshifts in the equilateral and squeezed configurations. However, the main idea is to train a Convolutional Neural Network (CNN) to classify the two cases, given snapshots of $\Phi$. The main analysis then involves interpretability of the CNN, which may be done by considering for instance saliency maps ~\parencite{alqaraawi_evaluating_2020} or Grad-CAM~\parencite{selvaraju_grad-cam_2020}. In either case, revealing the features separating the two cases may help us understand the differences in the gravitational dynamics between the two theories. I expect that such a network is able to find relativistic corrections to the Newtonian snapshots that are of higher order than those obtained from power spectra and bispectra analysis. Further, it may also reveal which configurations of Fourier modes $\vec{k}$ yield the highest bispectral power, which for now is mainly trial and error.  

% The whole dataset is used to train a Convolutional Neural Network (CNN), aiming to classify the two cases. If successful, the CNN may be used to analyse and understand the features separating the relativistic and Newtonian simulations. This is done through interpretability of the CNN by consider for instance saliency maps ~\parencite{alqaraawi_evaluating_2020} or Grad-CAM~\parencite{selvaraju_grad-cam_2020}. 