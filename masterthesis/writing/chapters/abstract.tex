On large scales, comparable to the horizon, various relativistic effect will affect the observable clustering properties of galaxies. In order to solve for these effects, one need to constantly solve for the metric, velocities and densities in a particular gauge. When simulating large-scale structures we often use N-body simulations, usually performed in the Newtonian limit. However, it is not obvious that Newtonian gravity yield a good global description of an in-homogeneous cosmology when there is significant nonlinear dynamical behaviour ~\parencite{jeong_large-scale_2012}. Literature results suggest that the relativistic corrections necessary on top of realistic Newtonian cosmologies should be very small~\parencite{chisari_connection_2011}. If this is the case, then this justifies the use of Newtonian simulations even on scales larger than the Hubble radius, whose results may be translated into relativistic cosmologies using relevant dictionaries~\parencite{green_newtonian_2012}.

I investigate this by running 2000 simulations with the relativistic N-body code \texttt{gevolution} by~\cite{adamek_gevolution_2016}, with and without relativistic effects, using identical $\Lambda$CDM cosmologies. The simulations are run on a $256^3$ grid each with dimension $5120\text{ Mpc}$ in order to capture the Hubble horizon. I investigate the difference between the gravities by considering the power spectra and bispectra of the gravitational potential $\Phi$, the latter should reveal the nonlinear dynamical behaviour present in the relativistic simulation. 

The whole dataset is used to train a Convolutional Neural Network (CNN), aiming to classify the two cases. If successful, the CNN may be used to analyse and understand the features separating the relativistic and Newtonian simulations. This may for instance be done using saliency maps. 