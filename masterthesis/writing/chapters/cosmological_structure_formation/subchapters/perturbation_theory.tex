%%%%%%%%%%%%%%%%%%%%%%
%
%   PERTURBATION THEORY
%
%%%%%%%%%%%%%%%%%%%%%%

\section{Initial Conditions}

\section{Transfer Functions}

\section{Power Spectra}
    

\section{Linear Evolution}

\section{Non-linear Evolution} 

\section{Bispectra} 

    The bispectra are powerful tools for studying the non-linear evolution of the density field. The bispectrum is defined as the Fourier transform of the three-point correlation function, and is given by:

    \begin{equation}
        \langle \delta(\vec{k}_1) \delta(\mathbf{k}_2) \delta(\mathbf{k}_3) \rangle = (2\pi)^3 \delta_D\left(\sum_i\vec{k}_i\right) B(\mathbf{k}_1, \mathbf{k}_2, \mathbf{k}_3)
    \end{equation}


    \subsection{Analytical Bispectum}
        \begin{equation}
            B^{(3)}_\delta(\vec{k}_1,\vec{k}_2,\vec{k}_3) = 2\mathcal{P}_\delta(k_1)\mathcal{P}_\delta(k_2)F_2(\vec{k}_1, \vec{k}_2) + \mathrm{cyc} = 2\epsilon_{ijk}F_2(\vec{k}_j,\vec{k}_k)\mathcal{P}_\delta(k_j)\mathcal{P}_\delta(k_k)
        \end{equation}

        \begin{equation}
            F_2(\vec{k}_1,\vec{k}_2) = \frac{5}{7} + \frac{x}{2}\left(\frac{k_1}{k_2}+\frac{k_2}{k_1}\right) + \frac{2}{7}x^2,
        \end{equation}
        where $x = \hat{\vec{k}}_1 \cdot \hat{\vec{k}}_2 = \cos{\theta_{12}}$, where $\theta_{12}$ is the angle spanned by $\vec{k}_1$ and $\vec{k}_2$. We could thus consequently write: $F_2(\vec{k}_1,\vec{k_2}) = F_2(k_1,k_2,\theta_{12})$.

        \paragraph{Work with kernel}
        \begin{figure}
            \centering
            \begin{tikzpicture}
    \colorlet{veccol}{green!70!black}
    \colorlet{vcol}{green!70!black}
    \colorlet{xcol}{blue!85!black}
    \colorlet{projcol}{xcol!60}
    \colorlet{unitcol}{xcol!60!black!85}
    \colorlet{myblue}{blue!70!black}
    \colorlet{myred}{red!90!black}
    \colorlet{mypurple}{blue!50!red!80!black!80}
    \tikzstyle{vector}=[->,very thick,xcol]
  
  
    % Define the k vector
    \def\Kone{5}
    \def\Ktwo{5}
    \def\Kthree{5}
    \def\staplelength{1}
    \def\anglealpha{60}
    \def\anglebeta{60}
    \def\anglegamma{60}
  
    \coordinate (k1start) at (0,0);
    \coordinate (k2start) at (\anglegamma:\Kone);
    \coordinate (k3start) at (\Kthree,0);
  
    % \Draw vector
    \draw[vector] (k1start) -- (k2start) node[midway, left=20, above=5, right=0] {$\vb{k}_1$};
    \draw[vector] (k2start) -- (k3start) node[midway, right=20, above=5, left=0] {$\vb{k}_2$};
    \draw[vector] (k3start) -- (k1start) node[midway, below=5] {$\vb{k}_3$};
    
    
    % coordinates for the extra stapled lines
    \coordinate (k1staplestop) at ({\Kone*cos(\anglegamma)+\staplelength*cos(\anglegamma)}, {\Kone*sin(\anglegamma)+\staplelength*sin(\anglegamma)});
    \coordinate (k2staplestop) at ({\Kthree+\staplelength*cos(\anglebeta)}, {-\staplelength*sin(\anglebeta)});
    \coordinate (k3staplestop) at ({-\staplelength}, 0);
  
    \draw[dashed] (k2start) -- (k1staplestop);
    \draw[dashed] (k3start) -- (k2staplestop);
    \draw[dashed] (k1start) -- (k3staplestop);
  
    \draw pic[-, thick, "$\alpha$", draw=black, angle radius = 25, angle eccentricity=1.3]{angle = k1start--k2start--k3start};
    \draw pic[-, thick, "$\beta$", draw=black, angle radius = 25, angle eccentricity=1.3]{angle = k2start--k3start--k1start};
    \draw pic[-, thick, "$\gamma$", draw=black, angle radius = 25, angle eccentricity=1.3]{angle = k3start--k1start--k2start};
  
    \draw pic[-, thick, "$\theta_{12}$", draw=black, angle radius = 15, angle eccentricity=1.6]{angle = k3start--k2start--k1staplestop};
    \draw pic[-, thick, "$\theta_{23}$", draw=black, angle radius = 15, angle eccentricity=1.6]{angle = k1start--k3start--k2staplestop};
    \draw pic[-, thick, "$\theta_{31}$", draw=black, angle radius = 15, angle eccentricity=1.6]{angle = k2start--k1start--k3staplestop};
  \end{tikzpicture}
            \caption{Angles in arbitrary bispectrum triangle configuration where $\sum_i \vec{k}_i=0$.}
            \label{fig:data:verification:bispectrum_angles}
          \end{figure}
        Given $k_1$ and $k_2$ and $\theta_{12}$ we have the following relations, with reference to \cref{fig:data:verification:bispectrum_angles}:

        \begin{equation}
            \begin{split}
            \alpha &= \pi-\theta_{12}\\
            \beta &= \pi-\theta_{23}\\
            \gamma &= \pi-\theta_{31}
            \end{split}
        \end{equation} 

        From cosine rule:
        \begin{equation}
            k_3 = \sqrt{k_1^2 + k_2^2 - 2k_1k_2\cos\alpha}
        \end{equation}

        From the rule of sines \TODO{explain more?}:
        \begin{equation}
            \begin{split}
            \beta &= \arcsin\left(\frac{k_1}{k_3}\sin\alpha\right)\\
            \gamma &= \arcsin\left(\frac{k_2}{k_3}\sin\alpha\right)
            \end{split}
        \end{equation}

        \paragraph{Bispectrum of potential}
            Turn this into the bispectrum of the potential, and then use the Poisson equation to get the bispectrum of the density field. Start with the Poisson equation (at late times), valid for all scales as long as $\delta_\mathrm{m}$ is given in synchronous gauge:
            \begin{equation}
                \begin{split}
                    k^2\Phi(\vec{k},a) &= 4\pi G a^2 \rho_\mathrm{m}(a) \delta_\mathrm{m}(\vec{k},a)\\
                    \Phi(\vec{k}, a) &= \frac{3}{2}\Omega_\mathrm{m} H_0^2 \frac{\delta_\mathrm{m}(\vec{k},a)}{ak^2} = \mathcal{C}(k,a)\delta_\mathrm{m}(\vec{k},a)
                \end{split}
            \end{equation}
            where in the last step I used that $\rho_\mathrm{m}(a) = \Omega_\mathrm{m} \rho_\mathrm{crit} a^{-3}$ and $8\pi G \rho_\mathrm{crit} = 3H_0^2$. I also defined the factor $\mathcal{C}(k, a)$ as follows:
        
            \begin{equation}
                \mathcal{C}(k,a) \equiv \frac{3\Omega_\mathrm{m}}{2a}\left(\frac{H_0}{k}\right)^2,
            \end{equation}
            where $H_0/k=1/(2997.13\cdot k)$ when $k$ is given in units of $h/\mathrm{Mpc}$. \TODO{explain this more?} This ensures that $\mathcal{C}(k,a)$ is dimensionless. The bispectrum of the potential is then given by:

            \begin{equation}
                \begin{split}
                    \langle \prod_{i=1}^3\Phi(\vec{k}_i,a) \rangle &= \langle \prod_{i=1}^3\mathcal{C}(k_i,a)\delta_\mathrm{m}(\vec{k}_i,a) \rangle \\
                    B_\Phi(\vec{k}_1,\vec{k}_2,\vec{k}_3, a) &=  B_\delta(\vec{k}_1,\vec{k}_2,\vec{k}_3, a) \prod_{i=1}^3\mathcal{C}(k_i,a). 
                \end{split}
            \end{equation}

            % \begin{equation}
            %     \begin{split}
            %         \langle \Phi(\vec{k}_1,a) \Phi(\vec{k}_2,a) \Phi(\vec{k}_3,a) \rangle &= \frac{\mathcal{C}(a)^3}{k_1^2k_2^2k_3^2} \langle \delta_\mathrm{m}(\vec{k}_1,a) \delta_\mathrm{m}(\vec{k}_2,a) \delta_\mathrm{m}(\vec{k}_3,a) \rangle\\
            %         (2\pi)^3 \delta_D\left(\sum_i\vec{k}_i\right) B_\Phi(\vec{k}_1,\vec{k}_2,\vec{k}_3)&= \frac{\mathcal{C}(a)^3}{k_1^2k_2^2k_3^2} (2\pi)^3 \delta_D\left(\sum_i\vec{k}_i\right) B_\delta(\vec{k}_1,\vec{k}_2,\vec{k}_3)\\
            %         B_\Phi(\vec{k}_1,\vec{k}_2,\vec{k}_3) &= \frac{\mathcal{C}(a)^3}{k_1^2k_2^2k_3^2} B_\delta(\vec{k}_1,\vec{k}_2,\vec{k}_3)
            %     \end{split}
            % \end{equation}
            % \TODO{fix some stuff with $a$ above}
            % Which leads to:

            % \begin{equation}
            %     B^{(3)}_\Phi(\vec{k}_1,\vec{k}_2,\vec{k}_3) = \frac{\mathcal{C}(a)^3}{k_1^2k_2^2k_3^2} \left[2\mathcal{P}_\delta(k_1)\mathcal{P}_\delta(k_2)F_2(\vec{k}_1, \vec{k}_2) + \mathrm{cyc}\right]
            % \end{equation}

            Using the same logic I find a relation between the power spectrum of the gravitational potential and the matter contrast:
            \begin{equation}
                \mathcal{P}_\Phi(k,a) = \mathcal{C}^2(k, a) \mathcal{P}_\delta(k,a) \Longleftrightarrow \mathcal{P}_\delta(k,a) = \mathcal{C}^{-2}(k, a) \mathcal{P}_\Phi(k,a)
            \end{equation}

            which results in the following analytical bispectrum for the gravitational potential:\footnote{Still only valid for even permutations of the Levi-Civita symbol of course.}

            % \begin{equation}
            %     B^{(3)}_\Phi(\vec{k}_1,\vec{k}_2,\vec{k}_3) = \frac{\mathcal{C}(a)^{-1}}{k_1^2k_2^2k_3^2} \left[2\mathcal{P}_\Phi(k_1)\mathcal{P}_\Phi(k_2)\tilde{F}_2(\vec{k}_1, \vec{k}_2) + \mathrm{cyc}\right]
            % \end{equation}

            % where the modified $F_2$-kerner is given by:

            % \begin{equation}
            %     \tilde{F}_2(\vec{k}_1, \vec{k}_2) \equiv k_1^4k_2^4\left[\frac{5}{7} + \frac{x}{2}\left(\frac{k_1}{k_2}+\frac{k_2}{k_1}\right) + \frac{2}{7}x^2\right]
            % \end{equation}

            \begin{equation}\label{}
                B^{(3)}_\Phi(\vec{k}_1,\vec{k}_2,\vec{k}_3, a) = \prod_{n=1}^3 \mathcal{C}(k_n, a) \left[\epsilon_{ijk}F_2(\vec{k}_j,\vec{k}_k)\prod_{l\in\{j,k\}}\mathcal{C}^{-2}(k_l,a)\mathcal{P}_\Phi(k_l,a)\right]
            \end{equation}




