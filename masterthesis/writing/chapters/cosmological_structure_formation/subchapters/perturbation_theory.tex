%%%%%%%%%%%%%%%%%%%%%%
%
%   PERTURBATION THEORY
%
%%%%%%%%%%%%%%%%%%%%%%

\section{Initial Conditions}

\section{Transfer Functions}

\section{Power Spectra}

\section{Non-linear Evolution} 

\section{Bispectra}

The bispectra are powerful tools for studying the non-linear evolution of the density field. The bispectrum is defined as the Fourier transform of the three-point correlation function, and is given by:

\begin{equation}
    \langle \delta(\vec{k}_1) \delta(\mathbf{k}_2) \delta(\mathbf{k}_3) \rangle = (2\pi)^3 \delta_D(\mathbf{k}_1 + \mathbf{k}_2 + \mathbf{k}_3) B(\mathbf{k}_1, \mathbf{k}_2, \mathbf{k}_3)
\end{equation}

Well this is rather awkward. \cite{adamek_gevolution_2016} or \autocite{falck_effect_2017}